% Компилировать xelatex'ом!
\documentclass[a4paper,12pt]{article}
\usepackage[english,russian]{babel}
\usepackage{amsmath, enumerate, multicol, listings}
\usepackage{xunicode, xltxtra, xecyr}
\setmainfont{Droid Serif}
\setmonofont{Droid Sans Mono}

% Поля
\usepackage{geometry}
\geometry{left=2cm}
\geometry{right=1.5cm}
\geometry{top=1cm}
\geometry{bottom=2cm}

% Полезности
\input{stuff}


\begin{document}

    % Заголовок
    \labtitle{Лабораторная работа №5}{Изучение работы команд Linux для работы с файлами}
  
    % Описание
    \begin{flushleft}
        Комманды Windows для работы с файлами:
        \begin{itemize}
            \item \cmd{chattr} - изменение аттрибутов файла
            \item \cmd{cd} - переход в заданную или в домашнюю папку
            \item \cmd{pwd} - вывод имени текущего каталога
            \item \cmd{cp} - копирование одного или нескольких файлов и каталогов
            \item \cmd{rm} - удаление файлов и каталогов
            \item \cmd{ls} - вывод списка файлов и подкаталогов каталога
            \item \cmd{cat} - вывод содержимого заданного файла на экран
            \item \cmd{diff}, \cmd{colordiff} - сравнение двух файлов и вывод различий между ними
            \item \cmd{grep} - поиск заданной строки текста в файле или нескольких файлах, в том числе и с использованием регулярных выражений
            \item \cmd{mkdir} - создание папки
            \item \cmd{touch} - создание файла
            \item \cmd{mv} - перемещение или переименование файлов и каталогов
        \end{itemize}
    \end{flushleft}
  
    \begin{flushleft}
        Для выполнения лабораторной работы необходимо запустить командную строку. Сделать это можно двумя способами:
        \begin{enumerate}
            \item В главном меню в подменю \cmd{Стандартные} выбрать \cmd{Эмулятор терминала}
            \item В контекстном меню рабочего стола выбрать \cmd{Открыть терминал}
        \end{enumerate}
    \end{flushleft}
  
    \begin{flushleft}
        Для получения более подробной информации по командам можно использовать команду \cmd{man} или запускать выбранную команду с ключом \cmd{-h}, например \cmd{cp -h}
    \end{flushleft}
  
    \begin{flushleft}
        Для работы с группой файлов в командной строке можно использовать маску фильтр. Символ \texttt{"?"} в имени файла означает, что в этом месте может быть любой символ, например \texttt{"lab?.cpp"} соответствует \texttt{"lab1.cpp"} и \texttt{"lab2.cpp"}, но уже не соответствует \texttt{"lab12.cpp"}. Символ \texttt{"*"} заменяет любое количество символов, например маске \texttt{"lab*.cpp"} соответствуют все вышеназванные примеры.
    \end{flushleft}

    \newpage

    \begin{flushleft}
        \center{\Large{Практические задания}} \\[0.5em]
        \begin{enumerate}
            \item Запустите Linux
            \item Составьте справочник для вышеприведенных команд (на русском языке), расписав какие параметры для чего нужны
            \item Поработайте с этими командами
            \item Что нужно уметь:
            \begin{enumerate} [\bf a. ]
                \item Запустить командную строку
                \item Вывести имя текущего каталога
                \begin{flushleft}
                    \cmd {pwd}
                \end{flushleft}
                \item Перейти в домашнюю папку
                \begin{flushleft}
                    \cmd {cd}
                \end{flushleft}
                \item Вывести список файлов и папок домашней директории
                \begin{flushleft}
                    \cmd {ls}
                \end{flushleft}
                \item Создать папку \texttt{test} в домашней папке
                \begin{flushleft}
                    \cmd {mkdir test}
                \end{flushleft}
                \item Перейти в папку \texttt{test}
                \begin{flushleft}
                    \cmd {cd test}
                \end{flushleft}
                \item Скопировать все файлы из \texttt{/usr/share/common-licenses} в \texttt{test}
                \begin{flushleft}
                    \cmd {cp /usr/share/common-licenses/* .}
                \end{flushleft}
                \item Отобразить скопированные файлы с информацией о размере и правах доступа
                \begin{flushleft}
                    \cmd {ls -lh}
                \end{flushleft}
                \item Переименовать файл \texttt{BSD} в \texttt{TEST}
                \begin{flushleft}
                    \cmd {mv BSD TEST}
                \end{flushleft}
                \item Удалить файл \texttt{TEST}
                \begin{flushleft}
                    \cmd {rm TEST}
                \end{flushleft}
                \item Найти номера строк в файле \texttt{GPL-1} содержащие текст \texttt{"the"}
                \begin{flushleft}
                    \cmd {grep -n „the“ GPL-1}
                \end{flushleft}
                \item Скопировать файлы с расширением \texttt{txt} в папку \texttt{test} из \texttt{/usr/share/vim/vim73/doc} с добавлением новых файлов
                \begin{flushleft}
                    \cmd {cp -u /usr/share/vim/vim73/doc/*.txt .}
                \end{flushleft}
                \item Сравнить файлы \texttt{GPL-1} и \texttt{GPL-2}
                \begin{flushleft}
                    \cmd {diff GPL-1 GPL-2}\\
                    или\\
                    \cmd {colordiff GPL-1 GPL-2}\\
                    \textit{(если выводит \cmd {bash: colordiff: command not found}, выполнить \cmd {sudo apt-get install colordiff})}
                \end{flushleft}
                \item Создать каталог \texttt{proba} в папке \texttt{test}
                \begin{flushleft}
                    \cmd {mkdir proba}
                \end{flushleft}
                \item Переместить файлы из папки \texttt{test} в \texttt{test/proba}
                \begin{flushleft}
                    \cmd {mv * proba/}\\
                    \textit{После выполнения выведет предупреждение: \cmd {mv: cannot move 'proba' to a subdirectory of itself, 'proba/proba'}. Его нужно проигнорировать.}
                \end{flushleft}
                \item Удалить файлы из \texttt{test/proba}
                \begin{flushleft}
                    \cmd {rm -f proba/*}
                \end{flushleft}
                \item Удалить каталог \texttt{proba}
                \begin{flushleft}
                    \cmd {rm -r proba}
                \end{flushleft}
                \item Скопировать все файлы и папки из \texttt{/usr/share/vim/vim73} в \texttt{test}
                \begin{flushleft}
                    \cmd {cp -r /usr/share/vim/vim73/* .}
                \end{flushleft}
                \item Удалить папку \texttt{test} включая все ее файлы и подкаталоги
                \begin{flushleft}
                    \cmd {cd .. \&\& rm -r test/}
                \end{flushleft}
            \end{enumerate}
        \end{enumerate}
    \end{flushleft}

\end{document}

