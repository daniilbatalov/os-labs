% Компилировать xelatex'ом!
\documentclass[a4paper,12pt]{article}
\usepackage[english,russian]{babel}
\usepackage{amsmath, enumerate, multicol, listings}
\usepackage{xunicode, xltxtra, xecyr}
\setmainfont{Droid Serif}
\setmonofont{Droid Sans Mono}

% Поля
\usepackage{geometry}
\geometry{left=2cm}
\geometry{right=1.5cm}
\geometry{top=1cm}
\geometry{bottom=2cm}

% Полезности
\input{stuff}


\begin{document}

  % Заголовок
  \labtitle{Лабораторная работа №2}{Управление процессами}
  
  % Описание
  \begin{flushleft}
    Команды Windows для работы с процессами:
    \begin{itemize}
     \item {\bf SCHTASKS} - настраивает выполнение команд по расписанию.
     \item {\bf START} - запускает определенную программу или команду в отдельном окне.
     \item {\bf TASKKILL} - завершает процесс.
     \item {\bf TASKLIST} - выводит информацию о работающих процессах.
    \end{itemize}
  \end{flushleft}
  

  % Вводная
  \begin{flushleft}
    Для выполнения лабораторной работы необходимо запустить командную строку, это можно сделать двумя способами:
    \begin{flushleft}
        \begin{enumerate} [1. ]
        \item Пуск -> Выполнить -> {\bf cmd}
        \item Пуск -> Программы -> Стандартные -> {\bf «Командная строка»} \\[0.4cm]
        \end{enumerate}
    \end{flushleft}
  \end{flushleft}

  \begin{flushleft}
  Для получения более подробной информации, можно использовать центр справки и поддержки Windows или консольную справку по команде с помощью ключа {\bf /?}, например, для просмотра справки по команде {\bf SCHTASKS} необходимо выполнить следующую команду {\bf SCHTASKS /?}\\[0.2cm]
  \end{flushleft}

  \newpage

  \begin{center}
    {\bf Практические задания}
  \end{center}

  \begin{flushleft}
    \begin{enumerate} [1. ]
     \item Запустите Windows
     \item Запустить командную строку
     \item Составьте справочник для выше приведенных команд(на русском языке), расписав какие параметры для чего нужны.
     \item Что нужно уметь:
      \begin{enumerate} [\bf a. ]
        \item Создать задание на запуск процесса в определённое время, Например: создать задание на запуск программы БЛОКНОТ в определённое время
        \begin{flushleft}
          \cmd {schtasks /Create /SC ONCE /ST "время" /TN "название задания" /TR notepad.exe}
        \end{flushleft}
        \item Отобразить список заданий на выполнение по времени, список задач отобразить в табличном и списочном варианте.
        \begin{flushleft}
          \cmd {schtasks /Query /TN "название задание" /FO LIST} для вывода в списочном варианте\\
          \cmd {schtasks /Query /TN "название задание"} для вывода в табличном варианте
        \end{flushleft}
        \item Изменить программу, которая должна быть выполнена в определённое время, например вместо программы БЛОКНОТ будет выполняться программа КАЛЬКУЛЯТОР
        \begin{flushleft}
          \cmd {schtasks /Change /TN "название задания" /TR calc.exe}
        \end{flushleft}
        \item Удалить задание
        \begin{flushleft}
         \cmd {schtasks /Delete /TN "название задания"}
        \end{flushleft}
        \item Запустить программу в отдельном окне в различных режимах, например, запустить программу БЛОКНОТ в развернутом окне и в свернутом окне, и с ожидание его завершения
        \begin{flushleft}
          \cmd {start /MIN notepad.exe} - в свернутом окне\\
          \cmd {start /MAX notepad.exe} - в развернутом окне\\
          \cmd {start /WAIT notepad.exe} - с ожиданием завершения
        \end{flushleft}
        \item Отобразить список процессов выполняющихся на рабочей станции
        \begin{flushleft}
         \cmd {tasklist}
        \end{flushleft}
        \item Отфильтровать список процессов так чтобы в нем остались только запущенные нами программы БЛОКНОТ в предыдущем задании
        \begin{flushleft}
         \cmd {tasklist /FI "IMAGENAME eq notepad.exe"}
        \end{flushleft}
        \item Отобразить пользователя, который запустил на выполнение эти процессы
        \begin{flushleft}
         \cmd {tasklist /FI "IMAGENAME eq notepad.exe" /V}
        \end{flushleft}
        \item Завершить процесс по номеру PID (идентификатор процесса)
        \begin{flushleft}
         \cmd {taskkill /PID "pid"}
        \end{flushleft}
        \item Завершить процесс по имени образа процесса, например, завершить все программы NOTEPAD.EXE 
        \begin{flushleft}
         \cmd {taskkill /FI "IMAGENAME eq notepad.exe"} \\ [1.5cm]
        \end{flushleft}
        \end{enumerate}
    \end{enumerate}

  \end{flushleft}

\end{document}
